\documentclass[a4paper, 11pt, titlepage]{article}

\usepackage[utf8]{inputenc}
\usepackage[german]{babel}
\usepackage{graphicx}
\usepackage[colorlinks,linkcolor=black,linktoc=all]{hyperref}

\author{Niko Lockenvitz, Edwin Brüseke,\\Manuel Bucher und Julius Bittner}
\title{\textsc{Nachhaltigkeit}\\\ \\\textsl{Einfacher lernen mit Rena}}

\begin{document}
\maketitle
\tableofcontents
\section{Rena}
Rena ist ein nachhaltiges Computerprogramm, mithilfe dessen man einfacher die Zuordnungen von Begriffen spielerisch erlernen kann. Es geht dabei in erster Linie um Vokabeln und ähnliches, beispielsweise die Zuordnungen von Bundesländern oder Staaten zu den zugehörigen Hauptstädten.\\\\
Der Name Rena rührt vom Aspekt der Nachhaltigkeit, er besteht aus dem Teil \textit{re}, der lateinischen Vorsilbe für \glqq zurück\grqq , und \textit{na}, dem Beginn des Worts \glqq Natur\grqq\ bzw. des lateinischen Worts \textit{natura}.\\\\
Neben den Aspekten der Nachhaltigkeit hat ein Computerprogramm die Vorteile, dass man es nicht betrügen kann, dementsprechend auch sich selbst beim Lernen nicht betrügen kann. Außerdem kann unser Programm eine Auswertung erstellen, ohne dass der Benutzer einen Befehl dazu geben muss. Man sieht als Schüler also auch, wie man sich verbessert hat, welches ungemein zur emotionalen Verbesserung des Schülers beiträgt.

\section{Bedienung}

\section{Aspekte der Nachhaltigkeit}
Viele Schüler haben Probleme beim Lernen. Unser Programm jedoch hilft dabei, Schüler beim Lernen zu unterstützen und somit mehr Freizeit zu geben, in denen sie \textendash\ solange der Wille besteht \textendash\ sich nachhaltig engagieren können.\\\\
Die Schüler haben, wenn sie ordentlich gelernt haben, im Endeffekt bessere Noten und dadurch bessere Perspektiven für ihre eigene Zukunft. Wer als erwachsener Mensch im Leben besser da steht, kann beispielsweise durch eine gute finanzielle Lage Spenden abgeben, um Menschen zu helfen, die das Geld dringend benötigen, es aber nicht selber aufwenden können.

\section{Arbeitseinteilung}

Für die Entwicklung unseres Programms haben wir die Arbeit wie folgt eingeteilt: Während Edwin die grafische Benutzeroberfläche eingerichtet hat, hat Niko die einzelnen Programmteile, die Manuel und Julius auf Nikos Beschreibung hin programmiert haben, sinnvoll zusammengestellt. Julius hat diesen Bericht zur Projektierung fertiggestellt. Letztendlich hat aber jeder an jedem Teil der Gestaltung des Programms mitgewirkt.\\\\
Zur einfacheren Bewertung zeigen wir in der folgenden Tabelle unsere individuellen Arbeitszeiten am Projekt auf:
\begin{table}[htbp]
\centering
\begin{tabular}{|c||c|c|c|c|}
\hline
Datum & Niko & Edwin & Manuel & Julius \\
\hline
\hline
05.12.2014 & 2 & 2 & 2 & 2 \\
\hline
06.12.2014 & 2 &  &  & 0 \\
\hline
07.12.2014 &  &  &  & 0 \\
\hline
08.12.2014 &  &  &  & 0 \\
\hline
09.12.2014 &  &  &  & 2 \\
\hline
10.12.2014 &  &  &  & \\
\hline
11.12.2014 &  &  &  & \\
\hline
12.12.2014 &  &  &  & \\
\hline
13.12.2014 &  &  &  & \\
\hline
14.12.2014 &  &  &  & \\
\hline
15.12.2014 &  &  &  & \\
\hline
16.12.2014 &  &  &  & \\
\hline
17.12.2014 &  &  &  & \\
\hline
\end{tabular}
\caption{Arbeitszeiten}
\end{table}

\end{document}
