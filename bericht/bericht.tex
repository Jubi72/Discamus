\documentclass[a4paper, 11pt, titlepage]{article}

\usepackage[utf8]{inputenc}
\usepackage{graphicx}
\usepackage[colorlinks,linkcolor=black,linktoc=all]{hyperref}

\author{Niko Lockenvitz, Edwin Brüseke,\\Manuel Bucher und Julius Bittner}
\title{\textsc{Nachhaltigkeit}\\\ \\\textsl{Einfacher lernen mit Discamus}}
\date{24. Februar 2015}

\begin{document}

\maketitle
\tableofcontents

\section{Discamus}
\textbf{Discamus} ist ein nachhaltiges Computerprogramm, mithilfe dessen man einfacher die Zuordnungen von Begriffen spielerisch erlernen kann. Es geht dabei in erster Linie um Vokabeln und Ähnliches, beispielsweise die Zuordnungen von Bundesländern oder Staaten zu den zugehörigen Hauptstädten.\\\\
Der Name Discamus rührt vom Inhalt des Programms. Discamus kommt aus dem Lateinischen und bedeutet \textit{Lasst uns lernen}, was sowohl einen positiven, motivierten Grundgedanken als auch einen Gemeinschaftsgedanken beinhaltet.\\\\
Neben den Aspekten der Nachhaltigkeit (s. u.) hat ein Computerprogramm die Vorteile, dass man es nicht betrügen kann, dementsprechend auch sich selbst beim Lernen nicht betrügen kann. Außerdem kann unser Programm eine Auswertung erstellen, ohne dass der Benutzer einen Befehl dazu geben muss. Man sieht als Schüler also auch, wie man sich verbessert hat, welches ungemein zur emotionalen Verbesserung des Schülers beiträgt. Selbstverständlich kann Discamus nicht nur von Schülern, sondern auch von anderen Zielgruppen genutzt werden.\\\\
Die Bedienung erläutern wir im Anhang.

\section{Aspekte der Nachhaltigkeit}
Viele Schüler haben Probleme beim Lernen. Unser Programm jedoch hilft dabei, Schüler beim Lernen zu unterstützen und somit mehr Freizeit zu geben, in denen sie \textendash\ solange der Wille besteht \textendash\ sich nachhaltig engagieren können. Weiterhin ist das Arbeitsmittel Computer ein weiterer Ansporn für junge Menschen, zu lernen, da Computer im Gegensatz zu Büchern als nicht mehr modern gelten und nicht so gerne benutzt werden.\\\\
Die Schüler haben, wenn sie ordentlich gelernt haben, im Endeffekt bessere Noten und dadurch bessere Perspektiven für ihre eigene Zukunft. Wer als erwachsener Mensch im Leben besser da steht, kann beispielsweise durch eine gute finanzielle Lage Spenden abgeben, um Menschen zu helfen, die das Geld dringend benötigen, es aber nicht selber aufwenden können.\\\\
Im Gegensatz zu herkömmlichen Lernmethoden bietet Discamus weiterhin den Vorteil, dass ein Computer im Gegensatz zum Menschen kein Nachsehen hat und der Anwender keine Möglichkeit zum Schummeln hat. Außerdem kann unser Lernprogramm die Eingaben auswerten und langfristig die Veränderungen speichern und dem Anwender anzeigen, sodass eine Verbesserung ein weiterer Motivationsfaktor für Schüler sein kann. Nicht zuletzt nimmt das Lernen außerdem weniger Zeit in Anspruch und schützt die Umwelt, da weder Papier noch Tinte verbraucht oder für überflüssige Notizen verschwendet werden müssen.\\\\

\section{Arbeitseinteilung}

Für die Entwicklung unseres Programms haben wir die Arbeit wie folgt eingeteilt: Während Edwin die grafische Benutzeroberfläche eingerichtet und die Betriebsanweisung (Anhang) zum Programm geschrieben hat, hat Niko die einzelnen Programmteile, die Manuel auf Nikos Beschreibung hin programmiert hat, sinnvoll zusammengestellt. Julius hat diesen Bericht zur Projektierung fertiggestellt. Letztendlich hat aber jeder an jedem Teil der Gestaltung des Programms mitgewirkt.\\\\
Zur einfacheren Bewertung zeigen wir in der folgenden Tabelle unsere individuellen Arbeitszeiten am Projekt auf, die Zeiten sind in Stunden angegeben. Der hohe Unterschied in   der Gesamtzahl der Stunden kommt daher zustande, dass einige Arbeiten sich als schwieriger und langwieriger herausgestellt haben als ursprünglich vermutet, was vorher nicht abzuschätzen war. Es sind alle Arbeitstage angegeben; an fehlenden Tagen haben wir nur passiv über das Projekt nachgedacht.
\begin{table}[htbp]
\centering
\begin{tabular}{|c||c|c|c|c|}
\hline
Datum & Niko & Edwin & Manuel & Julius \\
\hline
\hline
05.12.2014 & 2 & 2 & 2 & 2 \\
\hline
06.12.2014 & 2 & 0 & 1 & 0 \\
\hline
07.12.2014 & 0 & 0 & 1 & 0 \\
\hline
08.12.2014 & 0 & 2 & 3 & 0 \\
\hline
09.12.2014 & 0 & 1 & 0{,}5 & 3 \\
\hline
10.12.2014 & 0{,}5 & 0 & 0 & 0{,}5 \\
\hline
11.12.2014 & 2 & 1 & 2 & 0 \\
\hline
12.12.2014 & 1 & 1 & 0 & 0 \\
\hline
13.12.2014 & 1 & 0 & 0 & 0 \\
\hline
14.12.2014 & 2 & 0 & 2 & 0 \\
\hline
15.12.2014 & 0 & 0 & 2 & 0 \\
\hline
16.12.2014 & 1 & 0 & 0{,}5 & 0{,}5 \\
\hline
17.12.2014 & 0 & 0 & 1{,}5 & 0{,}5 \\
\hline
19.12.2014 & 0 & 0 & 2 & 0 \\
\hline
20.12.2014 & 0 & 0 & 1 & 0 \\
\hline
21.12.2014 & 2 & 0 & 2 & 0 \\
\hline
30.12.2014 & 0 & 0 & 1 & 0 \\
\hline
01.01.2015 & 0 & 0 & 1 & 0 \\
\hline
02.01.2015 & 1 & 0 & 0 & 0 \\
\hline
03.01.2015 & 4 & 0 & 0 & 0 \\
\hline
04.01.2015 & 2{,}5 & 0 & 2 & 0 \\
\hline
06.01.2015 & 1 & 0 & 0 & 0 \\
\hline
09.01.2015 & 0 & 0 & 1 & 0 \\
\hline
10.01.2015 & 1{,}5 & 0 & 1 & 0 \\
\hline
11.01.2015 & 0{,}5 & 0 & 0 & 0 \\
\hline
17.01.2015 & 0 & 0 & 1 & 0 \\
\hline
19.01.2015 & 0 & 0 & 1 & 0 \\
\hline
20.01.2015 & 0 & 0 & 1 & 0 \\
\hline
23.02.2015 & 1 & 0 & 1 & 0 \\
\hline
24.02.2015 & 3 & 3 & 3 & 3 \\
\hline
\hline
gesamt & 28 & 10 & 33{,}5 & 10 \\
\hline
\end{tabular}
\caption{Arbeitszeiten}
\end{table}

\end{document}
